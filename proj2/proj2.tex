\documentclass[a4paper, 11pt, twocolumn]{article}
\usepackage[utf8]{inputenc}
\usepackage[a4paper, left=1.5cm, top=2.5cm, text={18cm, 25cm}]{geometry}
\usepackage{csquotes}
\usepackage[czech]{babel}
\usepackage[IL2]{fontenc}
\usepackage[dvipsnames]{xcolor}
\usepackage{verbatim}
\usepackage{blindtext}
\usepackage{times}
\usepackage{amsthm, amssymb, amsmath, amsfonts}
\usepackage{nameref}
\usepackage{setspace}
\usepackage{wasysym}
\usepackage[hidelinks]{hyperref}

\theoremstyle{definition}
\newtheorem{definition}{Definice}[]
\newtheorem{sentence}{Věta}[]

\author{Adam Múdry\\xmudry01@stud.fit.vutbr.cz}

\begin{document}

  \begin{titlepage}
    \begin{center}
      \renewcommand{\baselinestretch}{0.97}
      \Huge\textsc{Fakulta informačních technologií \\
        Vysoké učení technické v Brně} \\
        \vspace{\stretch{0.38}}

        \renewcommand{\baselinestretch}{1.02}

        \LARGE{
          Typografie a publikování{\,}--{\,}2. projekt \\
          Sazba dokumentů a matematických výrazů \\
        }
        \vspace{\stretch{0.622}
      }
      \renewcommand{\baselinestretch}{1}
    \end{center}
    {\Large{2021\hfill Adam Múdry (xmudry01)}}
  \end{titlepage}

  \section*{Úvod}
  \label{sec:uvod}
  V této úloze si vyzkoušíme sazbu titulní strany, matematických vzorců, prostředí 
  a~dalších textových struktur obvyklých pro technicky zaměřené texty 
  (například rovnice~\eqref{eq:1} nebo Definice~\ref{definicia1} na straně~\pageref{sec:uvod}). 
  Rovněž si vyzkoušíme pou\-žívání odkazů \verb|\ref| a \verb|\pageref|.
  
  Na titulní straně je využito sázení nadpisu podle optického středu s využitím zlatého řezu. 
  Tento postup byl probírán na přednášce. Dále je použito odřádkování se zadanou relativní 
  velikostí 0.4 em a 0.3 em.
  
  V případě, že budete potřebovat vyjádřit matematickou konstrukci nebo symbol a~nebude se 
  Vám dařit jej nalézt v samotném {\LaTeX}u, doporučuji prostudovat možnosti ba\-líku maker {\AmS}-{\LaTeX}.

  \section{Matematický text}

  Nejprve se podíváme na sázení matematických symbolů a~výrazů v plynulém textu včetně sazby 
  definic a vět s vy\-užitím balíku \texttt{amsthm}. Rovněž použijeme poznámku pod čarou s použitím 
  příkazu \verb|\footnote|. Někdy je vhodné použít konstrukci \verb|\mbox{}|, která říká, že text nemá být zalomen.

  \begin{definition}
    \label{definicia1}
    Rozšířený zásobníkový automat \emph{(RZA) je definován jako sedmice tvaru $A = (Q, \Sigma, \Gamma, \delta, q_0, Z_0, F)$, kde:}
    \renewcommand\labelitemi{$\bullet$}
    \begin{itemize}
      \item $Q$ \emph{je konečná množina} vnitřních (řídicích) stavů,
      \item $\Sigma$ \emph{je konečná} vstupní abeceda,
      \item $\Gamma$ \emph{je konečná} zásobníková abeceda,
      \item $\delta$ \emph{je} přechodová funkce $Q \times(\Sigma\cup\{\epsilon\})\times\Gamma^{\ast}$ $\rightarrow$ $2^{{Q \times\Gamma}^{\ast}}$,
      \item $q_0 \in Q$ \emph{je} počáteční stav, $Z_0 \in \Gamma$ \emph{je} startovací symbol zásobníku a $F \subseteq Q$ \emph{je množina} koncových stavů.
    \end{itemize}
  \end{definition}

  Nechť \emph{$P = (Q, \Sigma, \Gamma, \delta, q_0, Z_0, F)$} je rozšířený zásobníkový automat. 
  \emph{Konfigurací} nazveme trojici \emph{$(q, w, \alpha) \in Q \times \Sigma^{\ast} \times \Gamma^{\ast}$},
  kde $q$ je aktuální stav vnitřního řízení, $w$ je dosud nezpracovaná část vstupního řetězce
  a~$\alpha = {Z_{i_1}}{Z_{i_2}}\dots{Z_{i_k}}$ je obsah zásobníku\footnote[1]{${Z_{i_1}}$ je vrchol zásobníku}.

  \subsection{Podsekce obsahující větu a odkaz}

  \begin{definition}
    \label{definicia2}
    Řetězec $w$ nad abecedou $\Sigma$ je přijat RZA \emph{$A$~jestliže $(q_0, w, Z_0) \overset{\ast}{\underset{A}{\vdash}} (q_F, \epsilon, \gamma)$  
    pro nějaké $\gamma \in \Gamma^{\ast}$ a $q_F \in F$. Množinu $L(A) = \{w \mid w$ je přijat RZA $A\} \subseteq$ $\Sigma^{\ast}$~nazýváme} jazyk přijímaný RZA $A$.
  \end{definition}

  Nyní si vyzkoušíme sazbu vět a důkazů opět s použitím balíku \texttt{amsthm}.

  \begin{sentence}
    \emph{Třída jazyků, které jsou přijímány ZA, odpovídá} bezkontextovým jazykům.
  \end{sentence}

  \begin{proof}
    V důkaze vyjdeme z Definice~\ref{definicia1} a~\ref{definicia2}.
  \end{proof}

  \section{Rovnice a odkazy}

  Složitější matematické formulace sázíme mimo plynulý text.
  Lze umístit několik výrazů na jeden řádek, ale pak je třeba tyto vhodně oddělit, 
  například příkazem \verb|\quad|.
  
  $$
    \sqrt[i]{x_i^3} \text{\quad kde } x_i \text{ je } i\text{-té sudé číslo splňujúci\quad} x_i^{x_i^{i^2}+ 2} \leq y_i^{x_i^4}
  $$

  V rovnici \eqref{eq:1} jsou využity tři typy závorek s různou explicitně definovanou velikostí.
  
  \begin{eqnarray}
      \label{eq:1}
      x & = & \Bigg[\bigg\{\Big[a+b\Big]\ast c\bigg\}^d \oplus 2\Bigg]^{3/2}\\
      y & = & \lim\limits_{x \to \infty} \frac{\frac{1}{\log_{10} x}}{sin^2x+cos^2x} \nonumber
  \end{eqnarray}

  V této větě vidíme, jak vypadá implicitní vysázení limity $\lim_{n \to \infty} f(n)$ 
  v normálním odstavci textu. Podobně je to i s dalšími symboly jako $\prod_{i=1}^{n} 2^i$ či $\bigcap_{A\in \mathcal{B}}A$. 
  V~pří\-padě vzorců $\lim\limits_{n \to \infty} f(n)$ a $\overset{n}{\underset{i=1}\prod} 2^i$ 
  jsme si vynutili méně úspornou sazbu příkazem \verb|\limits|.

  \begin{equation}
    \int_{b}^{a} g(x)dx\ =\ - \int\limits_{a}^{b} f(x) dx
  \end{equation}

  \section{Matice}

  Pro sázení matic se velmi často používá prostředí \texttt{array} a závorky (\verb|\left|, \verb|\right|).

  \begin{equation*}
    \left(\!
      \begin{array}{ccc}
        a - b  & \widehat{\xi + \omega} & \pi \\[0.3em]
        \vec{\mathbf{a}} & \overleftrightarrow{AC} & \hat{\beta}
      \end{array}
    \!\right)= 1 \Longleftrightarrow \mathcal{Q} =\mathbb{R}
  \end{equation*}

  \begin{equation*}
    \textbf{A}
    \ = \
    \begin{array}{||cccc||}
      \ a_{11} & a_{12} & \cdots & a_{1n}\ \\
      \ a_{21} & a_{22} & \cdots & a_{2n}\ \\
      \ \vdots & \vdots & \ddots & \vdots\ \\
      \ a_{m1} & a_{m2} & \cdots & a_{mn}\
    \end{array}
    \ = \
    \begin{array}{|cc|}
      \ t & u \ \\
      \ v & w \
    \end{array}\ =tw \minus uv
  \end{equation*}

  Prostředí \texttt{array} lze úspešně využít i jinde.

    \begin{equation*}
      \begin{pmatrix}
          n \\
          k 
      \end{pmatrix}
      =
      \begin{cases}
        \ \hfil 0 & \text{pro } k < 0 \text{ nebo } k > n \\
        \ \frac{n!}{k!(n-k)!} & \text{pro } 0\leq k\leq n\text{.} 
      \end{cases}
    \end{equation*}

\end{document}